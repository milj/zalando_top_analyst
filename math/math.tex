\documentclass[a4paper,12pt]{article}
\usepackage[a4paper]{geometry}
\usepackage[utf8]{inputenc}
\usepackage{indentfirst}
\usepackage{mathtools}
\usepackage{secdot}
\sectiondot{subsection}
\frenchspacing
\title{\vspace{-2cm}The mathematics of Zalando's Next Top Analyst problem}
\date{}
\author{Miłosz Jerkiewicz}
\begin{document}
\maketitle
\thispagestyle{empty}

This document contains some notes and mathematical work necessary to compute the probability distributions featuring
in the problem.


\section{Coordinate projection}

The goal of the problem is to find the geographical coordinates \((\mbox{latitude}, \mbox{longitude})\) of the high-probability areas
and present them in the form of an easy to read map.
Because the probability distributions used to evaluate \((\mbox{latitude}, \mbox{longitude})\) points have their parameters
specified in meters and they assume an orthogonal coordinate system, it is necessary to use a projection from one system
to another. The following projection is provided in the problem statement:

\begin{align*}
  x(\mbox{lat}, \mbox{lon}) &= 111.323\:(\mbox{lon} - \mbox{sw\_lon})\:\cos(\mbox{sw\_lat}\,\frac{\pi}{180})\ [km]\\
  y(\mbox{lat}, \mbox{lon}) &= 111.323\:(\mbox{lat} - \mbox{sw\_lat})\ [km]\\
\end{align*}

Where
\begin{align*}
  \mbox{sw\_lat} &= 52.464011\\
  \mbox{sw\_lon} &= 13.274099
\end{align*}

are the coordinates of the south-west point of the area of interest.

This projection returns \((X, Y)\) coordinates specified in kilometers. This is a slight inconvenience, because the probability distributions
have parameters specified im meters. By multiplying both equations by \(1000\) we get the final, ready-to-use equations:

\begin{align*}
  x(\mbox{lat}, \mbox{lon}) &= 111323\:(\mbox{lon} - \mbox{sw\_lon})\:\cos(\mbox{sw\_lat}\,\frac{\pi}{180})\ [m]\\
  y(\mbox{lat}, \mbox{lon}) &= 111323\:(\mbox{lat} - \mbox{sw\_lat})\ [m]
\end{align*}

\section{Log-normal distribution}

The probability density function of the log-normal distribution is given by the following formula:

\begin{equation*}
f(x) = \frac{1}{x\sigma{\sqrt{2\pi}}}\: e^{-{\frac {\left(\ln x-\mu \right)^{2}}{2\sigma ^{2}}}}
\end{equation*}

To compute \(f(x)\) at a given point \(x\), we need to know \(\mu\) and \(\sigma\). To compute them, we can use the following information:

\begin{quotation}
\emph{The distribution's radial profile is log-normal with a mean of 4700m and a mode of 3877m in every direction.}
\end{quotation}

The starting point are the formulas for the mean and the mode of the log-normal distribution:
\begin{align}
  \mbox{mean} &= e\, ^{\mu + \frac{\sigma^2}{2}} \label{eq:mean}\\
  \mbox{mode} &= e\, ^{\mu - \sigma^2} \label{eq:mode}
\end{align}

We need to solve this set of equations to get \(\mu\) and \(\sigma\). Logarithming both sides gives:
\begin{align*}
  \mu + \frac{\sigma^2}{2} &= \ln(\mbox{mean}) \tag{\ref{eq:mean}'}\\
  \mu - \sigma^2 &= \ln(\mbox{mode}) \tag{\ref{eq:mode}'}
\end{align*}

Substracting (\ref{eq:mode}') from (\ref{eq:mean}')
\begin{align*}
  \mu + \frac{\sigma^2}{2} - (\mu - \sigma^2) &= \ln(\mbox{mean}) - \ln(\mbox{mode})\\
  \frac{3}{2}\sigma^2 &= \ln(\mbox{mean}) - \ln(\mbox{mode})
\end{align*}

And finally:
\[\sigma = \sqrt{\frac{2}{3}\bigl(\ln(\mbox{mean}) - \ln(\mbox{mode})\bigr)}\tag{\(\sigma\)}\]

Getting back to equation (\ref{eq:mode}'):
\begin{align*}
  \mu - \sigma^2 &= \ln(\mbox{mode}) \tag{\ref{eq:mode}'}\\
  \mu &= \ln(\mbox{mode}) + \sigma^2
\end{align*}

Plugging the \(\sigma\) computed above:
\begin{align*}
  \mu &= \ln(\mbox{mode}) + \frac{2}{3}\left(\ln(\mbox{mean}) - \ln(\mbox{mode})\right)\\
  \mu &= \ln(\mbox{mode}) + \frac{2}{3}\ln(\mbox{mean}) - \frac{2}{3}\ln(\mbox{mode})\\
  \mu &= \frac{2}{3}\ln(\mbox{mean}) + \frac{1}{3}\ln(\mbox{mode})
\end{align*}

And finally:
\[\mu = \frac{2\ln(\mbox{mean}) + \ln(\mbox{mode})}{3}\tag{\(\mu\)}\]

The end result is the following set of formulas for \(\mu\) and \(\sigma\):
\begin{align*}
  \mu &= \frac{2\ln(\mbox{mean}) + \ln(\mbox{mode})}{3}\tag{\(\mu\)}\\
  \sigma &= \sqrt{\frac{2}{3}\left(\ln(\mbox{mean}) - \ln(\mbox{mode})\right)\tag{\(\sigma\)}}
\end{align*}

For the given data (\emph{a mean of 4700m and a mode of 3877m}) the values are:
\begin{align*}
  \mu &= 8.39115083\\
  \sigma &= 0.358237212
\end{align*}

\end{document}
